%----macros begin-----------------------------------------------------------------------------------
%\usepackage{amsmath}
\usepackage{graphicx}
\usepackage{color}
\usepackage{xcolor}
\usepackage{amsfonts}
%\usepackage{amsthm}
%\usepackage[utf8]{inputenc}
\usepackage{framed}
\usepackage{wasysym}

\newcommand{\columnsbegin}{\begin{columns}}
\newcommand{\columnsend}{\end{columns}}

%\renewenvironment{Shaded}{\pause\begin{snugshade}}{\end{snugshade}}
\def\twocolumns#1#2{\begin{columns}
\begin{column}{0.5\linewidth}#1\end{column}
\begin{column}{0.5\linewidth}#2\end{column}
\end{columns}}
\def\mytwocolumns#1#2#3#4{\begin{columns}
\begin{column}{#1\linewidth}#2\end{column}
\begin{column}{#3\linewidth}#4\end{column}
\end{columns}}
\def\mythreecolumns#1#2#3#4#5#6{\begin{columns}
\begin{column}{#1\linewidth}#2\end{column}
\begin{column}{#3\linewidth}#4\end{column}
\begin{column}{#5\linewidth}#6\end{column}
\end{columns}}
\def\threecolumns#1#2#3{\begin{columns}
\begin{column}{0.33\linewidth}#1\end{column}
\begin{column}{0.33\linewidth}#2\end{column}
\begin{column}{0.33\linewidth}#3\end{column}
\end{columns}}
\def\fourcolumns#1#2#3#4{\begin{columns}%
\begin{column}{0.25\linewidth}#1\end{column}%
\begin{column}{0.25\linewidth}#2\end{column}%
\begin{column}{0.25\linewidth}#3\end{column}%
\begin{column}{0.25\linewidth}#4\end{column}%
\end{columns}}

\def\textbf#1{\alert{#1}}
\def\emph#1{{\color{cyan}#1}}
\def\conv{\mbox{\textrm{conv}\,}}
\def\aff{\mbox{\textrm{aff}\,}}
\def\E{\mathbb{E}}
\def\R{\mathbb{R}}
\def\Z{\mathbb{Z}}
\def\N{\mathbb{N}}
\def\P{\mathbb{P}}
\def\v#1{{\bf #1}}
\def\p#1{{\bf #1}}
\def\T#1{{\bf #1}}
\def\vet#1{{\left(\begin{array}{cccccccccccccccccccc}#1\end{array}\right)}}
\def\mat#1{{\left(\begin{array}{cccccccccccccccccccccccccccc}#1\end{array}\right)}}

\def\lin{\mbox{\rm lin}\,}
\def\aff{\mbox{\rm aff}\,}
\def\pos{\mbox{\rm pos}\,}
\def\cone{\mbox{\rm cone}\,}
\def\conv{\mbox{\rm conv}\,}
\newcommand{\homog}[0]{\mbox{\rm homog}\,}
\newcommand{\relint}[0]{\mbox{\rm relint}\,}

\newtheorem{assignment}{Assignment}
\newtheorem{exercise}{Exercise}
\newtheorem{question}{Question}
\newtheorem{remark}{Remark}
%----macros end-----------------------------------------------------------------------------------

\newcommand\myenum[1]{% 
  \begin{pgfpicture}{-1ex}{-0.65ex}{1ex}{1ex} 
    \usebeamercolor[fg]{item projected} 
    {\pgftransformscale{1.75}\pgftext{\normalsize\pgfuseshading{bigsphere}}} 
    {\pgftransformshift{\pgfpoint{0pt}{0.5pt}} 
      \pgftext{\usebeamerfont*{item projected}#1}} 
  \end{pgfpicture}% 
} 

\definecolor{whitesmoke}{HTML}{F5F5F5}

%
%\usepackage{listings}
%\lstdefinestyle{bash-style}{
%  captionpos=b,
%  belowcaptionskip=1\baselineskip,
%  breaklines=true,
%  tabsize=2,
%  frame=tb,
%  aboveskip=3mm,
%  belowskip=3mm,
%  xleftmargin=\parindent,
%  language=bash,
%  showstringspaces=false,
%  % basicstyle=\tiny, %\footnotesize\ttfamily,
%  basicstyle={\fontsize{8pt}{8pt}\ttfamily},
%  keywordstyle=\color{black},
%  commentstyle=\color{green!40!black},
%  stringstyle=\color{brown},
%  identifierstyle=\color{black},
%  backgroundcolor=\color{whitesmoke}
%}
%\lstdefinestyle{python-style}{
%  captionpos=b,
%  belowcaptionskip=1\baselineskip,
%  breaklines=true,
%  tabsize=2,
%  frame=tb,
%  aboveskip=3mm,
%  belowskip=3mm,
%  xleftmargin=\parindent,
%  language=Python,
%  showstringspaces=false,
%  % basicstyle=\tiny, %\footnotesize\ttfamily,
%  basicstyle={\fontsize{8pt}{8pt}\ttfamily},
%  keywordstyle=\color{blue},
%  commentstyle=\color{green!40!black},
%  stringstyle=\color{brown},
%  identifierstyle=\color{blue},
%  backgroundcolor=\color{whitesmoke}
%}


% Default fixed font does not support bold face
\DeclareFixedFont{\ttb}{T1}{txtt}{bx}{n}{12} % for bold
\DeclareFixedFont{\ttm}{T1}{txtt}{m}{n}{12}  % for normal

% Custom colors
\usepackage{color}
\definecolor{deepblue}{rgb}{0,0,0.5}
\definecolor{deepred}{rgb}{0.6,0,0}
\definecolor{deepgreen}{rgb}{0,0.5,0}
\definecolor{whitesmoke}{HTML}{F5F5F5}

\usepackage{listings}

% Python style for highlighting
\newcommand\pythonstyle{\lstset{
language=Python,
basicstyle=\ttm,
otherkeywords={self},             % Add keywords here
keywordstyle=\ttb\color{orange},
emph={MyClass,__init__},          % Custom highlighting
emphstyle=\ttb\color{deepred},    % Custom highlighting style
stringstyle=\color{deepgreen},
frame=tb,                         % Any extra options here
showstringspaces=false,           % 
  captionpos=b,
  belowcaptionskip=1\baselineskip,
  breaklines=true,
  tabsize=2,
  frame=tb,
  aboveskip=3mm,
  belowskip=3mm,
  xleftmargin=\parindent,
  language=Python,
  showstringspaces=false,
  basicstyle=\footnotesize\ttfamily,
  basicstyle={\fontsize{8pt}{8pt}\ttfamily},
  % keywordstyle=\color{blue},
  commentstyle=\color{deepgreen}\slshape,
  stringstyle=\color{brown},
  identifierstyle=\color{blue},
% language=python,
% basicstyle=\ttfamily\scriptsize\setstretch{1},
% stringstyle=\color{red},
% showstringspaces=false,
% alsoletter={1234567890},
% otherkeywords={\ , \}, \{},
% keywordstyle=\color{blue},
emph={access,and,break,class,continue,def,del,elif ,else,%
except,exec,finally,for,from,global,if,import,in,i s,%
lambda,not,or,pass,print,raise,return,try,while},
emphstyle=\color{black}\bfseries,
emph={[2]True, False, None, self},
emphstyle=[2]\color{green},
emph={[3]from, import, as},
emphstyle=[3]\color{red},
upquote=true,
morecomment=[s]{"""}{"""},
%commentstyle=\color{gray}\slshape,
emph={[4]1, 2, 3, 4, 5, 6, 7, 8, 9, 0},
% emphstyle=[4]\color{blue},
%literate=*{:}{{\textcolor{blue}:}}{1}%
%{=}{{\textcolor{blue}=}}{1}%
%{-}{{\textcolor{blue}-}}{1}%
%{+}{{\textcolor{blue}+}}{1}%
%{*}{{\textcolor{blue}*}}{1}%
%{!}{{\textcolor{blue}!}}{1}%
%{(}{{\textcolor{blue}(}}{1}%
%{)}{{\textcolor{blue})}}{1}%
%{[}{{\textcolor{blue}[}}{1}%
%{]}{{\textcolor{blue}]}}{1}%
%{<}{{\textcolor{blue}<}}{1}%
%{>}{{\textcolor{blue}>}}{1},%  
  backgroundcolor=\color{whitesmoke}
}}


% Python environment
\lstnewenvironment{python}[1][]
{
\pythonstyle
\lstset{#1}
}
{}

% Python for external files
\newcommand\pythonexternal[2][]{{
\pythonstyle
\lstinputlisting[#1]{#2}}}

% Python for inline
\newcommand\pythoninline[1]{{\pythonstyle\lstinline!#1!}}


